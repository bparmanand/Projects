%\documentclass[11pt]{paper}
%\usepackage{fullpage}
%\usepackage{palatino}
%\usepackage{amsfonts,amsmath,amssymb}
%% \usepackage{graphicx}
%
%\usepackage{listings}
%\usepackage{textcomp}
%\usepackage{color}
%
%\definecolor{dkgreen}{rgb}{0,0.6,0}
%\definecolor{gray}{rgb}{0.5,0.5,0.5}
%\definecolor{mauve}{rgb}{0.58,0,0.82}
%
%\lstset{frame=tb,
%  language=R,
%  aboveskip=3mm,
%  belowskip=3mm,
%  showstringspaces=false,
%  columns=flexible,
%  basicstyle={\small\ttfamily},
%  numbers=none,
%  numberstyle=\tiny\color{gray},
%  keywordstyle=\color{blue},
%  commentstyle=\color{dkgreen},
%  stringstyle=\color{mauve},
%  breaklines=true,
%  breakatwhitespace=true,
%  tabsize=3
%}
%
%
%
%\ifx\pdftexversion\undefined
%    \usepackage[dvips]{graphicx}
%\else
%    \usepackage[pdftex]{graphicx}
%    \usepackage{epstopdf}
%    \epstopdfsetup{suffix=}
%\fi
%
%\usepackage{subfig}
%
%
%% This allows pdflatex to print the curly quotes in the
%% significance codes in the output of the GAM.
%\UseRawInputEncoding
%
%\begin{document}
%
%%%%%%%%%%%%%%%%%%%%%%%%%%%%%%%%%%%%%%%%%
%% Problem Set 7
%%%%%%%%%%%%%%%%%%%%%%%%%%%%%%%%%%%%%%%%%
%
%\pagestyle{empty}
%{\noindent\bf Spring 2023 \hfill Brandon~Parmanand}
%\vskip 16pt
%\centerline{\bf University of Central Florida}
%\centerline{\bf College of Business}
%\vskip 16pt
%\centerline{\bf QMB 6911}
%\centerline{\bf Capstone Project in Business Analytics}
%\vskip 10pt
%\centerline{\bf Solutions:  Problem Set \#7}
%\vskip 32pt
%\noindent
%% 
%\section{Data Description}
%
%This analysis follows the script \texttt{PS7.R} to produce a more accurate model for used tractor prices with the data from \texttt{homesales.dat} in the \texttt{Data} folder. 

I will revisit the recommended linear model.
I will further investigate nonlinear relationships
by incorporating another nonlinear but parametric specification
for variables.
This parametric analysis will be performed
using the Box-Tidwell framework
to investigate whether the value of these characteristics
are best described with parametric nonlinear forms. 

%%%%%%%%%%%%%%%%%%%%%%%%%%%%%%%%%%%%%%%%
\section{Linear Regression Model}
%%%%%%%%%%%%%%%%%%%%%%%%%%%%%%%%%%%%%%%%

\subsection{Suggested Linear Regression Model}
A natural staring point is the recommended linear model
from previous which we can see in Table \ref{tab:reg_comp}. 

% 

\begin{table}
\begin{center}
\begin{tabular}{l c}
\hline
 & Model 1 \\
\hline
(Intercept)       & $12.2109^{***}$ \\
                  & $(0.0333)$      \\
TypeOfBuyerRental & $0.1126^{***}$  \\
                  & $(0.0160)$      \\
Age               & $-0.0108^{***}$ \\
                  & $(0.0004)$      \\
NumBeds2          & $0.2209^{***}$  \\
                  & $(0.0260)$      \\
NumBeds3          & $0.4205^{***}$  \\
                  & $(0.0314)$      \\
NumBeds4          & $0.6106^{***}$  \\
                  & $(0.0346)$      \\
NumBeds6          & $0.7039^{***}$  \\
                  & $(0.0562)$      \\
NumBeds8          & $1.0794^{***}$  \\
                  & $(0.0612)$      \\
NumBaths2         & $0.0721^{***}$  \\
                  & $(0.0139)$      \\
NumBaths3         & $0.1033^{***}$  \\
                  & $(0.0271)$      \\
LotSize           & $0.0000^{***}$  \\
                  & $(0.0000)$      \\
HasGarage         & $0.2638^{***}$  \\
                  & $(0.0194)$      \\
HasSecGate        & $0.2238^{***}$  \\
                  & $(0.0180)$      \\
TransitScore      & $0.0636^{***}$  \\
                  & $(0.0030)$      \\
\hline
R$^2$             & $0.6745$        \\
Adj. R$^2$        & $0.6728$        \\
Num. obs.         & $2473$          \\
\hline
\multicolumn{2}{l}{\scriptsize{$^{***}p<0.001$; $^{**}p<0.01$; $^{*}p<0.05$}}
\end{tabular}
\caption{Models for the Log. of House Sales}
\label{tab:reg_comp}
\end{center}
\end{table}

% 
There are houses that were built in 2021 so in order to account for this in the model, an edit was made to calculate age by taking 2022- Year Built. Additionally, Number of Beds and Numbers of Bathrooms were treated as factors in the model as they more so categorize the houses.
%
\clearpage


%%%%%%%%%%%%%%%%%%%%%%%%%%%%%%%%%%%%%%%%
% \clearpage
\section{Nonlinear Specifications}
%%%%%%%%%%%%%%%%%%%%%%%%%%%%%%%%%%%%%%%%

%\pagebreak
\subsection{The Box--Tidwell Transformation}

The Box--Tidwell function tests for non-linear relationships
to the mean of the dependent variable.
The nonlinearity is in the form of an
exponential transformation in the form of the Box-Cox
transformation, except that the transformation is taken
on the explanatory variables.


\subsubsection{Transformation of Lot Size}


Performing the transformation on the Lot Size variable
produces a modified form of the linear model. The exponentis significantly differnt from 0. WItrh a small positive value that suggest an increasing relationship then leveling off to a slower increase. 

\begin{verbatim} MLE of lambda Score Statistic (z)  Pr(>|z|)    
       -0.2736             -3.6889 0.0002252 ***
---
Signif. codes:  0 �***� 0.001 �**� 0.01 �*� 0.05 �.� 0.1 � � 1

iterations =  3 
\end{verbatim}



\subsubsection{Transformation of Age}


\begin{verbatim} MLE of lambda Score Statistic (z)  Pr(>|z|)    
       0.49715              6.4677 9.949e-11 ***
---
Signif. codes:  0 �***� 0.001 �**� 0.01 �*� 0.05 �.� 0.1 � � 1

iterations =  3 
\end{verbatim}

This coefficient is significantly different than 0. With a positive value that suggest not quite a linear relationship but it is an increasing relationship with it leveling off and increasing at a slower rate.  

\subsubsection{Transformation of Transit Score}


\begin{verbatim} MLE of lambda Score Statistic (z) Pr(>|z|)
        1.1833              1.0571   0.2905

iterations =  3 
\end{verbatim}

The transit score coefficient is not statistically significant but it is close to 1 showing close to linear realationship.

Since a nonlinear relationship was detected with age and lot size,
I will next estimate a model
with nonlinearity in all three continuous variables.


\subsubsection{Transformation of All Three Continuous Variables}


\begin{verbatim}             MLE of lambda Score Statistic (z)  Pr(>|z|)    
TransitScore       1.20854              1.2318 0.2180097    
Age                0.50460              6.4167 1.392e-10 ***
LotSize           -0.27686             -3.6641 0.0002482 ***
---
Signif. codes:  0 �***� 0.001 �**� 0.01 �*� 0.05 �.� 0.1 � � 1

iterations =  3 
\end{verbatim}


The performance is similar to the other models with
forms of nonlinearity .



\pagebreak
\section{Linear Approximation of the Box--Tidwell Transformation}

I created three variables 
\texttt{bt\_trans\_log\_trans}, \texttt{bt\_age\_log\_age}, and \texttt{bt\_lot\_log\_lot}, 
all of which were created by a transformation of the form $f(x) = x\cdot\log(x)$. 
Table \ref{tab:reg_bt_lin} collects the results
of the set of models from the nonlinear approximation to the models with the three forms of nonlinearity.
Model 1 is the linear regression model with  
the approximation of the transformation applied to transit score. 
Models 2 and 3
have the same specification as the other one, 
except that the transit score variable is replaced with
the variables for age and lot size, respectively. 
The coefficient on \texttt{bt\_age\_log\_age}
is the most statistically significant. 
And the coefficient on \texttt{bt\_lot\_log\_lot} is also statistically signifcant. 
This implies, just as the Box-Tidwell statistic predicts, 
a nonlinear relationship exists for the value of age and lot size.
The transit score is not statistically significant as the other two
indicating that a linear relationship suffices for the decline in value from transit score.


\begin{table}
\begin{center}
\begin{tabular}{l c c c}
\hline
 & Model 1 & Model 2 & Model 3 \\
\hline
(Intercept)           & $12.25507^{***}$ & $12.27956^{***}$ & $11.89089^{***}$ \\
                      & $(0.05344)$      & $(0.03468)$      & $(0.09288)$      \\
TypeOfBuyerRental     & $0.10997^{***}$  & $0.11437^{***}$  & $0.11362^{***}$  \\
                      & $(0.01621)$      & $(0.01590)$      & $(0.01599)$      \\
Age                   & $-0.01083^{***}$ & $-0.03428^{***}$ & $-0.01089^{***}$ \\
                      & $(0.00042)$      & $(0.00365)$      & $(0.00042)$      \\
NumBeds2              & $0.22205^{***}$  & $0.22090^{***}$  & $0.20754^{***}$  \\
                      & $(0.02600)$      & $(0.02576)$      & $(0.02616)$      \\
NumBeds3              & $0.41967^{***}$  & $0.41973^{***}$  & $0.40137^{***}$  \\
                      & $(0.03141)$      & $(0.03114)$      & $(0.03174)$      \\
NumBeds4              & $0.59709^{***}$  & $0.60678^{***}$  & $0.59744^{***}$  \\
                      & $(0.03693)$      & $(0.03437)$      & $(0.03474)$      \\
NumBeds6              & $0.68745^{***}$  & $0.69817^{***}$  & $0.72136^{***}$  \\
                      & $(0.05831)$      & $(0.05573)$      & $(0.05624)$      \\
NumBeds8              & $1.06090^{***}$  & $1.07728^{***}$  & $1.11651^{***}$  \\
                      & $(0.06363)$      & $(0.06068)$      & $(0.06185)$      \\
NumBaths2             & $0.07221^{***}$  & $0.07219^{***}$  & $0.07012^{***}$  \\
                      & $(0.01387)$      & $(0.01376)$      & $(0.01384)$      \\
NumBaths3             & $0.10338^{***}$  & $0.10791^{***}$  & $0.10408^{***}$  \\
                      & $(0.02715)$      & $(0.02694)$      & $(0.02708)$      \\
LotSize               & $0.00004^{***}$  & $0.00004^{***}$  & $0.00053^{***}$  \\
                      & $(0.00000)$      & $(0.00000)$      & $(0.00014)$      \\
HasGarage             & $0.26282^{***}$  & $0.26785^{***}$  & $0.24900^{***}$  \\
                      & $(0.01941)$      & $(0.01924)$      & $(0.01975)$      \\
HasSecGate            & $0.22309^{***}$  & $0.22254^{***}$  & $0.22569^{***}$  \\
                      & $(0.01804)$      & $(0.01788)$      & $(0.01798)$      \\
TransitScore          & $0.03918$        & $0.06332^{***}$  & $0.06433^{***}$  \\
                      & $(0.02329)$      & $(0.00296)$      & $(0.00298)$      \\
bt\_trans\_log\_trans & $0.00923$        &                  &                  \\
                      & $(0.00874)$      &                  &                  \\
bt\_age\_log\_age     &                  & $0.00608^{***}$  &                  \\
                      &                  & $(0.00094)$      &                  \\
bt\_lot\_log\_lot     &                  &                  & $-0.00005^{***}$ \\
                      &                  &                  & $(0.00001)$      \\
\hline
R$^2$                 & $0.67469$        & $0.67999$        & $0.67634$        \\
Adj. R$^2$            & $0.67284$        & $0.67817$        & $0.67449$        \\
Num. obs.             & $2473$           & $2473$           & $2473$           \\
\hline
\multicolumn{4}{l}{\scriptsize{$^{***}p<0.001$; $^{**}p<0.01$; $^{*}p<0.05$}}
\end{tabular}
\caption{Linear Approximation of Box-Tidwell Transformations for House Prices}
\label{tab:reg_bt_lin}
\end{center}
\end{table}



\pagebreak
\section{Comparison of Candidate Models}

Comparing the three models, Age and Lot Size were both significantly signifcant with p less than $0.001$ but the model with the age transformation had the highest R squared. I created a variable \texttt{age\_bt}
by raising age to the optimal exponent 
$\hat{\lambda} = 0.49715$. 
Then, I included this variable in the place of 
the age variables a the linear regression model.
% 
Table \ref{tab:reg_sq_horse_bt} collects the results
of the set of models from the two forms of nonlinearity.
Model 1 is the approximation to the Box-Tidwell transformation
from Model 2 of Table \ref{tab:reg_bt_lin}. 
Model 2
has the same specification as the approximate transformation, 
except that the age variable is transformed using the optimal
exponent for the Box-Tidwell transformation. 
% 
The last model has the highest R-squared
among the ones we have estimated, 
with only a slight improvement over the linear approximation.
Again, the differences are marginal, so the practical recommendation
is model with the $f(x) = x\cdot\log(x)$.



\begin{table}
\begin{center}
\begin{tabular}{l c c}
\hline
 & Model 1 & Model 2 \\
\hline
(Intercept)       & $12.27956^{***}$ & $12.34070^{***}$ \\
                  & $(0.03468)$      & $(0.03395)$      \\
TypeOfBuyerRental & $0.11437^{***}$  & $0.11404^{***}$  \\
                  & $(0.01590)$      & $(0.01589)$      \\
Age               & $-0.03428^{***}$ &                  \\
                  & $(0.00365)$      &                  \\
NumBeds2          & $0.22090^{***}$  & $0.22147^{***}$  \\
                  & $(0.02576)$      & $(0.02575)$      \\
NumBeds3          & $0.41973^{***}$  & $0.42032^{***}$  \\
                  & $(0.03114)$      & $(0.03113)$      \\
NumBeds4          & $0.60678^{***}$  & $0.60720^{***}$  \\
                  & $(0.03437)$      & $(0.03435)$      \\
NumBeds6          & $0.69817^{***}$  & $0.69834^{***}$  \\
                  & $(0.05573)$      & $(0.05571)$      \\
NumBeds8          & $1.07728^{***}$  & $1.07896^{***}$  \\
                  & $(0.06068)$      & $(0.06066)$      \\
NumBaths2         & $0.07219^{***}$  & $0.07189^{***}$  \\
                  & $(0.01376)$      & $(0.01375)$      \\
NumBaths3         & $0.10791^{***}$  & $0.10731^{***}$  \\
                  & $(0.02694)$      & $(0.02692)$      \\
LotSize           & $0.00004^{***}$  & $0.00004^{***}$  \\
                  & $(0.00000)$      & $(0.00000)$      \\
HasGarage         & $0.26785^{***}$  & $0.26760^{***}$  \\
                  & $(0.01924)$      & $(0.01923)$      \\
HasSecGate        & $0.22254^{***}$  & $0.22260^{***}$  \\
                  & $(0.01788)$      & $(0.01787)$      \\
TransitScore      & $0.06332^{***}$  & $0.06332^{***}$  \\
                  & $(0.00296)$      & $(0.00296)$      \\
bt\_age\_log\_age & $0.00608^{***}$  &                  \\
                  & $(0.00094)$      &                  \\
age\_bt           &                  & $-0.08524^{***}$ \\
                  &                  & $(0.00316)$      \\
\hline
R$^2$             & $0.67999$        & $0.68005$        \\
Adj. R$^2$        & $0.67817$        & $0.67836$        \\
Num. obs.         & $2473$           & $2473$           \\
\hline
\multicolumn{3}{l}{\scriptsize{$^{***}p<0.001$; $^{**}p<0.01$; $^{*}p<0.05$}}
\end{tabular}
\caption{Alternate Models for House Prices}
\label{tab:reg_sq_horse_bt}
\end{center}
\end{table}




%%%%%%%%%%%%%%%%%%%%%%%%%%%%%%%%%%%%%%%%
%\end{document}
%%%%%%%%%%%%%%%%%%%%%%%%%%%%%%%%%%%%%%%%
